\documentclass{article}
\title{Examples in mathematical mode in C4}
\author{Student}
\begin{document}
\maketitle
\section{A matrix}
%using array
Here begins a text in line, followed by a matrix in line,  
$A(x,y)=\left(
\begin{array}{ccc}
\frac{x}{y} & 0 & 0\\
1 & 0 & x
\end{array}
\right)$. After the matrix,  return into the text mode. But we can edit the same matrix on a separate line like this, with \verb+$$...$$+:
$$A(x,y)=\left(
\begin{array}{ccc}
\frac{x}{y} & 0 & 0\\
1 & 0 & x
\end{array}
\right)$$ or like this, with \verb+\[...\]+:
\[A(x,y)=\left(
\begin{array}{ccc}
\frac{x}{y} & 0 & 0\\
1 & 0 & x
\end{array}
\right)\]
\section{Several equations}
%using equation and eqnarray
Here are several equations, numbered on separate lines. The equation counter is a variable called equation.
\begin{equation}\label{ec:e}
\lim_{n\rightarrow\infty}\left(1+\frac{1}{n}\right)^{n}=e
\end{equation}
\begin{equation}\label{ec:lap}
F(s)=\int_{0}^{\infty}f(t)e^{-st}dt
\end{equation}
This is an example of the \verb+eqnarray+ environment, allowing to align the \verb+=+ sign of several equations separated by \verb+\\+. The \verb+\nonumber+ command is placed before the equation we do not want to be numbered.
\begin{eqnarray}
x_n&=&x_{n-1}+x_{n-2}\\
\nonumber
f(x)&=&x^2\\
\nonumber
\lefteqn{g(x)=x^2+y^2}\\
& &+z^2
\end{eqnarray}
If equation \ref{ec:lap} is displayed in line it looks like this: $F(s)=\int_{0}^{\infty}f(t)e^{-st}dt$.
%using equation and array
Here is an example of a function defined on intervals:
\begin{equation}
f(x)=\left\{
\begin{array}{ll}
x,&x>=0\\
-x,&x<0
\end{array}\right.
\end{equation}
Note how the \verb+array+ environment is delimited.
\end{document}